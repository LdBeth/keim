\documentstyle[11pt,code]{article}
%Kuerzel fuer Systeme
\def\keim{\mbox{${\cal KEIM}$}}
\newcommand{\mkrp}{{\sc mkrp}}
\newcommand{\omegamkrp}{$\Omega$-{\mkrp}}
\newcommand{\lisp}{{\sc Lisp}}
\newcommand{\commonlisp}{{\sc CommonLisp}}
\def\genera{{\sc Genera}}
\newcommand{\motiv}{{\sc Motif}}
\def\emacs{{\sc Emacs}}
\def\sccs{{\sc Sccs}}
\def\post{\POST}
\def\PostBox{{{\POST}.$\Box$}}
\def\POST{\mbox{${\cal POST}$}}
\def\clos{{\sc Clos}}

\def\ags{{AGS}}

%Tastatureingaben
\def\control{{\tt <control>}}
\def\esc{{\tt esc}}
\def\meta{{\tt <meta>}}
\def\super{{\tt <super>}}
\def\hyper{{\tt <hyper>}}
\def\rubout{{\tt <rubout>}}
\def\return{{\tt <cr>}}
\def\shift{{\tt <shift>}}
\def\abort{{\tt <abort>}}
\def\spacekey{{\tt <space>}}
\def\tab{{\tt <tab>}}
\def\endkey{{\tt <end>}}
\def\var#1{{\it #1}}\def\key#1{{\tt #1}}
\def\prog#1{{\tt #1}}\def\kbd#1{{\tt #1}}


\newlength{\argumentwidth}
\newlength{\minargwidth}
\setlength{\minargwidth}{15em}
\newlength{\namewidth}
\newlength{\typewidth}

\newenvironment{lispdefinition}[3]%
   {\goodbreak\removelastskip\bigskip
    \index{#2}
    \settowidth{\namewidth}{\bf #2}
    \settowidth{\typewidth}{[#1]}
    \setlength{\argumentwidth}{\textwidth}
    \addtolength{\argumentwidth}{-1\namewidth}
    \addtolength{\argumentwidth}{-1\typewidth}
    \addtolength{\argumentwidth}{-2em}
    \ifdim\argumentwidth<\minargwidth%
           \ifdim\argumentwidth<-1.5em%
                  \setlength{\argumentwidth}{\textwidth}%
                  \addtolength{\argumentwidth}{-7em}%
                  \begin{bf}#2\end{bf}\newline%
                  \hspace*{\fill}[#1]\newline%
                  \hspace*{\fill}\parbox[t]{\argumentwidth}{\raggedright #3}%
             \else%
                  \setlength{\argumentwidth}{\textwidth}%
                  \addtolength{\argumentwidth}{-7em}%
                  \begin{bf}#2\end{bf}\hspace{\fill}[#1]\newline%
                  \hspace*{\fill}\parbox[t]{\argumentwidth}{\raggedright #3}%
             \fi
       \else%
           \begin{bf}#2\end{bf}\hspace{1em}\parbox[t]{\argumentwidth}{\raggedright #3}\hspace{1em}[#1]%
       \fi
    \begin{lispdeclarations}}%
   {\end{lispdeclarations}
    \removelastskip\bigskip}

\newenvironment{lispdeclarations}%
   {\begin{list}%
           {}%
           {\itemsep1ex
            \topsep0ex
            \partopsep0mm
            \parsep0ex
            \leftmargin8em
            \rightmargin1.5em
            \labelwidth7em
            \labelsep1em}}%
   {\end{list}}

\newlength{\slotoptionwidth}
\newlength{\optionwidth}
\newenvironment{structure}[1]%
    {\parskip0ex
     \goodbreak
     \removelastskip
     \bigskip
     \begin{bf}#1\end{bf}\hspace*{\fill}[Structure]
     \newline}%
    {\removelastskip
     \bigskip}

\newenvironment{structureoptions}%
    {\hspace*{2em} options:
     \begin{list}%
            {}%
            {\topsep0ex
             \partopsep0ex
             \parsep0ex
             \itemsep0ex
             \settowidth{\labelwidth}{:additional-functions}
             \labelsep1em
             \setlength{\leftmargin}{\labelwidth}
             \addtolength{\leftmargin}{4em}
             \addtolength{\leftmargin}{\labelsep}
             \rightmargin3em
             \setlength{\optionwidth}{\textwidth}
             \addtolength{\optionwidth}{-\rightmargin}
             \addtolength{\optionwidth}{-\leftmargin}}}%
    {\end{list}}

\newenvironment{structureslots}%
    {\goodbreak
     \hspace*{2em} slots:
     \begin{list}%
            {}%
            {\topsep0ex
             \partopsep0ex
             \parsep0ex
             \itemsep0ex
             \setlength{\labelwidth}{10em}
             \labelsep1em
             \setlength{\leftmargin}{\labelwidth}
             \addtolength{\leftmargin}{4em}
             \addtolength{\leftmargin}{\labelsep}
             \rightmargin3em}}%
    {\end{list}}

\newenvironment{slotoptions}%
    {\setlength{\slotoptionwidth}{\textwidth}
     \addtolength{\slotoptionwidth}{-\rightmargin}
     \addtolength{\slotoptionwidth}{-\leftmargin}
     \begin{minipage}[t]{\slotoptionwidth}
          \raggedright
          \begin{list}%
                 {}%
                 {\settowidth{\labelwidth}{:parameter-access}
                  \labelsep1em
                  \setlength{\leftmargin}{\labelwidth}
                  \addtolength{\leftmargin}{\labelsep}
                  \rightmargin0em}}%
    {     \end{list}
     \end{minipage}}

\newenvironment{structuredocs}%
    {\goodbreak
     \hspace*{2em} documentation:\newline
     \addtolength{\textwidth}{-4em}
     \addtolength{\textwidth}{-\rightmargin}
     \hspace*{4em}\begin{minipage}[t]{\textwidth}}%
    {              \end{minipage}}



%%%%%%%%%%%%%%%%%%%%%%%%%%%%%%%%%%%%%%%%%%%%%%%%%%%%%%%%%%%%
%% Macros for POST syntax declarations
%%%%%%%%%%%%%%%%%%%%%%%%%%%%%%%%%%%%%%%%%%%%%%%%%%%%%%%%%%%%
%% These are the macros for documenting POST syntax.
%% use them in the lisp file in the form
%%
%%\begin{postsyntax}
%%\syntax{\nt{clause} ::= (clause \nt{head} \{\nt{literal}\}*).
%% \nt{ende}.}
%%\syntaxcomment{Some crazy comment goes here, but it is optional}
%%\syntax{\nt{END}.
%%\end{postsyntax}
%%
%% Here, if somewhere before there is an occurrnce of \makepostsyntax
%%  the argument in \syntax will be formated in a verbatim-like 
%% environment (code.sty) and will be written to a file ***.postsyntax, 
%% which will eventually contain all post syntax declarations.
%% This file can then be formatted into a section with the command 
%% \makepostsyntaxsection. 
%% However at the moment the linefeedss in the argument of \syntax
%% will be listed as ^^M in the ***.postsyntax file.
%%
%% The macro \postindex will write its argument into an index.
%%


\newif\ifpostsyntax
\postsyntaxfalse
\newwrite\postsyntaxfile
\def\makepostsyntaxsection{\ifpostsyntax\immediate\closeout\postsyntaxfile\else\fi
                           \section{Post Syntax}\label{sec:postsyntax}
                           \begin{code}
                           \input \jobname.postsyntax
                           \end{code}}

\def\postindex#1{\index{post syntax ! {#1}}}

\def\makepostsyntax{\postsyntaxtrue\immediate\openout\postsyntaxfile=\jobname.postsyntax}
\def\nt#1{{\it #1}}

\long\def\syntax{\catcode`\^^M=13\catcode`\ =13\long\def\p##1{\begin{code}##1\end{code}%
\def\it{\noexpand\it}\def\{{\noexpand\{}\def\}{\noexpand\}}%\catcode`\^=11\def\^^M{\noexpand\\}\catcode`\^=13%
\ifpostsyntax\immediate\write\postsyntaxfile{##1}\else\fi}\p}
\def\syntaxcomment{\catcode`\^^M=5\catcode`\ =10}
\newenvironment{postsyntax}{\bgroup}{\egroup}



%%%%%%%%%%%%%%%%%%%%%%%%%%%%%%%%%%%%%%%%%%%%%%%%%%%%%%%%%%%%%%%
%% Miscelaneous functions
%%%%%%%%%%%%%%%%%%%%%%%%%%%%%%%%%%%%%%%%%%%%%%%%%%%%%%%%%%%%%%%

\def\vb{\catcode`\~=12\tt}


\title{Using \protect{\keim}}
\author{Dan Nesmith}
\begin{document}

\maketitle

\section{Terms and Environments}

The way to enter a term in \keim\ is to use a specific notation called
\post.  We'll give some examples below, simplifying things of course to
get you off the ground.  
For more details see the
full \keim\ manual.  

\subsection{The basic ideas}

\post\ is a typed language.  That is, every term has a type, and every
term must be well-typed.  There are two ``standard'' types, {\tt O} and
{\tt I}, which stand for the types of propositions and individuals,
respectively.  Although these have something of a special status, you
still have to declare them in every environment.  More about that later.

Anyway, we have base or primitive types (such as {\tt O} and {\tt I}) and
we also have function types.  A function type encodes the range and domain
of a function.  For example, the function type {\tt (I I)} encodes the
type of a function which takes an individual as argument and returns an
individual as result.  The function type {\tt (I I I)} can be thought of
as the type of a function taking two arguments of type {\tt I}, and returning
a result of type {\tt I}.   If we use currified types, we would write this
{\tt ((I I) I)}, but we always associate the parentheses to the left and
omit as many as possible.  

So far we have been ambiguous.  To be more precise, a type is either a
\begin{code}
{\it type} ::= {\it basetype} | {\it functiontype}
{\it basetype} ::= {\it symbol}.
{\it functiontype} ::= ({\it range} {\it domain\(_n\)} \ldots {\it domain\(_1\)})
{\it range} ::= {\it type}.
{\it domain\(_i\)} ::= {\it type}.
\end{code}

The function type shown above denotes the function taking arguments (in order)
of type {\it domain$_1$} through {\it domain$_n$} and returning a value of
type {\it range}. 

So for another example, a relation on the individuals would have 
as type {\tt (O I I)}.

We also allow type variables.  A type variable can be used to denote, for
example, functions that could be thought of as polymorphic, that is, as 
being applied to more than one type.  Examples are ideas like union and
intersection of sets, as well as quantifiers (more below).

\subsection{Environments}

An environment is used (mostly) to provide a mapping between the external
representation of a symbol and its type, so that when we input the formula
we don't have to write its type with it.  That is, if the symbol {\tt X} is
already in the environment as a constant of type {\tt I}, we can just
write {\tt X} in our input form.  For new bound variables, we have to
write the type. 

We have to explicitly create environments and put our constants in them.
Creating an environment is done by {\vb (env~create)}.  

To tell the environment that something is a legal type symbol, we say either:

\begin{code}
(type-variables {\it symbol}*)
(type-constants {\it symbol}*)

Examples:
 (type-constants O I)
 (type-variables AA BB)
\end{code}

\subsection{Symbols}

Formulas are based on symbols.  You can think of a symbol as a name
(case-insensitive string) and a type.  We declare symbols to have a
particular type in an environment by the following forms.

\begin{code}
{\it declaration} ::= (constants {\it decl}+) |(variables {\it decl}+).
{\it decl} ::= ({\it name} {\it type}).
{\it name} ::= {\it symbol}.

Example: (constants (A I) (P (O I I)))
\end{code}

\subsection{Application}

One way to form terms from symbols is by application.  Application is
done merely by putting in parentheses first the function and then the
arguments.  The types of the function and arguments must match, however,
or an error will be signaled.  For example:

If {\tt A} has type {\tt I}, and {\tt P} has type {\tt (O I)},
 then {\tt (P A)} is an application
of type {\tt O}. 

But {\tt (P P)} is not a legal application, because {\tt P}'s argument must be of
type {\tt I}.


A special case is when some of the subterms involved have type variables
in their types.  Then a matching is done to make sure that the variables
can be instantiated with concrete types before allowing the term to be
created.

\subsection{Abstraction}

Abstraction allows us to bind variables.  Like other languages, we use
a lambda, but we write it {\tt LAM} for short.  {\tt LAM} is a special
symbol, and you should only use it for binding a variable.

If {\tt A} is a term of type {\tt T$_1$}, and {\tt T$_2$} is
another valid type, and {\it var} is any symbol name, 
then the term {\tt (lam ({\it var} T$_2$) A)} is
a term of type {\tt (T$_1$ T$_2$)}.   While the term is being created,
{\it var} is temporarily placed in the environment with type {\tt T$_2$},
and it shadows any occurrence of the same identifier that was already
in the environment.  That is, in {\vb (lam (X I) (lam (X I) (P X)))}
the argument of {\tt P} is the innermost bound occurrence of {\tt X}.
The resulting term is of type {\tt ((O I) I)} or to write it more simply
{\tt (O I I)}.

\subsection{Quantification}

We can quantify formulas by merely putting a predicate on a set.  For example,
when we interpret the formula $\forall x (P x)$, we are in effect saying 
that the set determined by taking the $x$'s for which $P x$ is true is
everything in the universe.  For $\exists$, it merely says that this set is nonempty.

For our quantification, we therefore 
take the quantifiers to be predicates on predicates, that is, a quantifier
takes a single-argument predicate (a function which returns a boolean as
value) and returns a boolean.  That is, a quantifier has the range type 
{\tt O}, and its domain type must be of type {\tt (O A)}, for some type
{\tt A}.  What is {\tt A}?  Well, just as we can think of quantifying
over all kinds of sets (numbers, people, etc), we want to apply our
quantifiers as widely as possible.  We let this {\tt A} be a type variable.

That is, we can declare our quantifiers by writing
\begin{code}
(constants (forall (O (O AA))) (exists (O (O AA))))
\end{code}
assuming that {\tt AA} has been declared as a type variable.

\subsection{Examples}

To actually get the terms to be read, we use the functions 
{\vb post~read-object-list} and {\vb term~read}.

\begin{code}
\small
(in-package :keim) ;make sure we're in the right package
(setq env (env~create)) ; we create a new environment
(post~read-object-list '((type-constants O I) (type-variables AA BB)) env)
; now we have some valid types in the environment,
; we want to declare some constants.
(post~read-object-list '((constants (X I) (P (O I)))) env)
(term~read '(P X) env) 
(term~read '(P P) env)  ; something illegal because ill-typed
(term~read '(P W) env)  ; something illegal because W was not declared
(term~read '(lam (Y I) (P X)) env)
; how about a quantifier?
(post~read-object-list '((constants (forall (O (O AA))) 
                                    (exists (O (O AA))))) env)	
(term~read '(forall (lam (Y I) (P X))) env)

\end{code}



\end{document}
