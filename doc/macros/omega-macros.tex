%***********************************************************
%    Makros fuer Omega-MKRP
%
%***********************************************************
\typeout{Omega-MKRP Makros Version 1.1 vom 11. Juni 1991 zuletzt ge"andert am 2.12.1992}
% Autoren M. Kohlhase, M. Kerber
% zuletzt geaendert am 2.12.1992, um laengere Assumptionlisten zu ermoegliche.
% Dafuer wurden die Macros \assref, \recref, \auxpl und \auxthmline geaendert.
% Es werden nun zusaetzliche Blanks hinter jedem , und ; geschrieben, so dass umgebrochen
% werden kann.

% Diese Theoremdefinitionen schreiben Definitionen Bemerkungen in 
% automatisch in \rm und nicht in \it wie normal. Sie benutzen aber
% die normalen LaTeX Numerierung und verhalten sich in allem anderen
% wie die normalen Umgebungen

\newtheorem{Thm}{Theorem}[section]
\newtheorem{Prop}[Thm]{Proposition}
\newtheorem{Cor}[Thm]{Corollary}
\newtheorem{Lemma}[Thm]{Lemma}
\newtheorem{iGenass}[Thm]{General assumption}
\newtheorem{iDef}[Thm]{Definition}
\newtheorem{iNotation}[Thm]{Notation}
\newtheorem{iRemark}[Thm]{Remark}
\newtheorem{iRemarks}[Thm]{Remarks}
\newtheorem{iExample}[Thm]{Example}
\newtheorem{iExamples}[Thm]{Examples}
\newtheorem{iAuxmethod}[Thm]{Method}

\makeatletter
\def\@opargbegindefinition#1#2#3{\trivlist
      \item[\hskip \labelsep{\bf #1\ #2\ (#3)}]\rm}
\def\@begindefinition#1#2{\it \trivlist \item[\hskip \labelsep{\bf #1\ #2}]\rm}
\def\Defcase{\let\@opargbegintheorem=\@opargbegindefinition\let\@begintheorem=\@begindefinition}
\makeatother

\newenvironment{Def}{\Defcase\begin{iDef}}{\end{iDef}}
\newenvironment{Notation}{\Defcase\begin{iNotation}}{\end{iNotation}}
\newenvironment{Rem}{\Defcase\begin{iRemark}}{\end{iRemark}}
\newenvironment{Rems}{\Defcase\begin{iRemarks}}{\end{iRemarks}}
\newenvironment{Ex}{\Defcase\begin{iExample}}{\end{iExample}}
\newenvironment{Exs}{\Defcase\begin{iExamples}}{\end{iExamples}}
\newenvironment{Genass}{\Defcase\begin{iGenass}}{\end{iGenass}}

\newenvironment{warning}{{\huge Warning!}\em}{\par}

% Voreinstellungen f"ur das Seitenformat
%Seitengroesse
\textwidth15.5cm \textheight23.2cm %23cm
%\textheight15.5cm \textwidth23.2cm %23cm
\oddsidemargin8mm \evensidemargin8mm 
\topmargin-10mm
\parindent0ex


\let\phi\varphi
\let\epsilon\varepsilon
\newcommand{\comment}[1]{}
\def\Dom{\hbox{\rm Dom}}

%   Logische Zeichen
\def\fol{\Rightarrow} % Logische Folgerung
\def\lof{\Leftarrow} % Umgekehrte logische Folgerung
\def\TO{\mbox{\tt ->}} % -> in Sorten
\def\lnot{\neg}
\def\sdot{\rule{2pt}{2pt}\hskip1pt}

%   Spezielle Zeichen
\def\phi{\varphi}
\def\cA{\mbox{$\cal A$}}\def\cB{\mbox{$\cal B$}}\def\cC{\mbox{$\cal C$}}
\def\cD{\mbox{$\cal D$}}\def\cE{\mbox{$\cal E$}}\def\cF{\mbox{$\cal F$}}
\def\cG{\mbox{$\cal G$}}\def\cH{\mbox{$\cal H$}}\def\cI{\mbox{$\cal I$}}
\def\cJ{\mbox{$\cal J$}}\def\cK{\mbox{$\cal K$}}\def\cL{\mbox{$\cal L$}}
\def\cM{\mbox{$\cal M$}}\def\cN{\mbox{$\cal N$}}\def\cO{\mbox{$\cal O$}}
\def\cP{\mbox{$\cal P$}}\def\cQ{\mbox{$\cal Q$}}\def\cR{\mbox{$\cal R$}}
\def\cS{\mbox{$\cal S$}}\def\cT{\mbox{$\cal T$}}\def\cU{\mbox{$\cal U$}}
\def\cV{\mbox{$\cal V$}}\def\cW{\mbox{$\cal W$}}\def\cX{\mbox{$\cal X$}}
\def\cY{\mbox{$\cal Y$}}\def\cZ{\mbox{$\cal Z$}}
\def\A{\mbox{\bf A}}\def\B{\mbox{\bf B}}
\def\C{\mbox{\bf C}}\def\D{\mbox{\bf D}}
\def\E{\mbox{\bf E}}\def\F{\mbox{\bf F}}
\def\G{\mbox{\bf G}}\def\H{\mbox{\bf H}}
\def\I{\mbox{\bf I}}\def\J{\mbox{\bf J}}
\def\K{\mbox{\bf K}}\def\L{\mbox{\bf L}}
\def\M{\mbox{\bf M}}\def\N{\mbox{\bf N}}
\def\O{\mbox{\bf O}}\def\P{\mbox{\bf P}}
\def\Q{\mbox{\bf Q}}\def\R{\mbox{\bf R}}
\def\S{\mbox{\bf S}}\def\T{\mbox{\bf T}}
\def\U{\mbox{\bf U}}\def\V{\mbox{\bf V}}
\def\W{\mbox{\bf W}}\def\X{\mbox{\bf X}}
\def\Y{\mbox{\bf Y}}\def\Z{\mbox{\bf Z}}

\def\NN{\hbox{\bf I\kern-0.2em N}}
\def\ZZ{\hbox{\bf Z\kern-0.4em Z}}
\def\RR{\hbox{\bf I\kern-0.2em R}}
\def\CC{\setbox0=\hbox{\bf C}\hskip0.2em\rule{1pt}{\ht0}\kern-0.4em \box0}
\def\QQ{\setbox0=\hbox{\bf Q}\hskip0.2em\rule{1pt}{\ht0}\kern-0.4em \box0}


%***************************************************************************
%   Hier kommen die Makros f"ur die Umgebung ndproof
%***************************************************************************
% 
% Die Umgebung ndproof und ndproofsection erwarten ein Argument, ein Label
% fuer die interne Beweisnummer des Beweises.
% ndproof setzt das label, mit dem dann ueber \ref zugegriffen werden kann.
% Mit ndproofsection kann
% dann ein Teil des Beweises mit dem richtigen Label ausgelagert werden,
% dabei werden alle Zeilen und labels im Kontext dieses Beweises referenziert.
% Der Benutzer muss dann selbst darauf achten, das die labelung konsistent
% bleibt.
% Die Umgebungen ndproof und  ndproofsection
% stellt die Makros \proofheader, \pl, \planline, \ndproofdevider und
% \thmline zur Verf"ugung.
%
% Das Makro \ndproofheader erzeugt eine Kopfzeile mit den Tabellenueber-
% schriften No, S;D, Formula, Reason
% 
% Das Makro \pl erzeugt eine Beweiszeile, die automatisch durchnumeriert
% wird, es hat 4 Argumente:
% <label>, <assump>, <formula>, <reason>; dabei ist <label> ein Name 
% fuer die Zeilenummer, mit dem unter <assump> und <reason> auf die Zeile 
% referenziert werden kann.
% <label> ist dabei eine beliebige Zeichenkette, die keine TeX-Sonder- 
%     Zeichen (\,%,&,$,...), aber auch nicht die Zeichen "," ";" "."
%     und "|" enthalten darf.
% <assump> ist eine durch Kommata getrennte Liste von vorher definierten
%     <labels> 
% <formula> ist die Formel, die in der Zeile stehen soll.
% <reason> ist die Begr"undung fuer die Beweiszeile, sie besteht aus
%     einer Regel, gefolgt duch ein Semikolon ";" gefolgt von einer
%     durch Kommata getrennten Liste von definierten labels.
%     ACHTUNG!! Wenn die Liste von labels leer ist muss immer noch ein
%     Semikolon stehen, wenn nicht: viel Spass beim Suchen in den
%     kryptischen Fehlermelldungen.
%
% \planline hat nur die Argumente 
% <label>, <formula> und erzeugt eine Beweiszeile, wo die
% Assumptionliste nur aus der Zeilennummer besteht und die Begruendung
% aus (PLAN).
% 
% \thmline erzeugt die abschliessende Theoremzeile eines Beweises und
% hat die Argumente <assump>, <formula>, <reason>, da auf diese Zeile
% nicht mehr referenziert werden muss
%
% \ndproofdevider hat ein Argument und zieht einen waagerechten
% Strich durch den Beweis zu Verschoenerungszwecken und schreibt das
% Argument in die Mitte
%
% \pletc erzeugt eine Beweiszeile, in der in jeder Spalte vertikale
% unsoweiter Punkte stehen
%
% \nextlinenumber gibt der folgenden Beweiszeile die Nummer die im Argument
% steht. Ab da wird dann auch automatisch weiternumeriert.
%************************************************************************

\newcounter{plineno}

\def\doref#1{\ref{\ndprooflabel:#1}}

\def\recref#1,#2.{{%
\def\tist{#1}\def\test{#2}%
\ifx\tist\empty{}\else\recorref #1|.\fi%
%\ifx\test\empty{}\else,\recref #2.\fi}}
\ifx\test\empty{}\else, \recref #2.\fi}}

\def\recorref#1|#2.{{%
\def\tist{#1}\def\test{#2}%
\ifx\tist\empty{}\else\doref{#1}\fi%
\ifx\test\empty{}\else\big|\recorref #2.\fi}}

\def\rulerecref#1,#2.{{%
\def\tist{#1}\def\test{#2}%
\ifx\tist\empty{}\else\ \doref{#1}\fi%
\ifx\test\empty{}\else\ \rulerecref #2.\fi}}

\def\ruleref#1;#2.{#1\rulerecref #2,.}

%\def\assref#1;#2.{{\recref #1,.};{\recref #2,.}}
\def\assref#1;#2.{{\recref #1,.}; {\recref #2,.}}

\newcommand{\auxpl}[3]%assump,formula,reason
{\arabic{plineno}. 
%\> \parbox[t]{2.4cm}{\assref #1.}\> $\vdash$ \>
\> \begin{minipage}[t]{2.4cm}\raggedright{\assref #1.}\end{minipage}\> $\vdash$ \>
\begin{minipage}[t]{8.4cm}\raggedright{#2}\end{minipage}
\> \parbox[t]{3cm}{(\ruleref #3.)}\\}

\newcommand{\auxthmline}[3]%assump,formula,reason
{Thm.
%\> \parbox[t]{2.4cm}{\assref #1.}\> $\vdash$ \>
\> \begin{minipage}[t]{2.4cm}\raggedright{\assref #1.}\end{minipage}\> $\vdash$ \>
\begin{minipage}[t]{9.4cm}\raggedright{#2}\end{minipage}
\> \parbox[t]{2cm}{(\ruleref #3.)}\\}

\newcommand{\dolabel}[1]{\label{#1}}

\newcommand{\pl}[4]%label,assump,formula,reason
{\refstepcounter{plineno}\dolabel{\ndprooflabel:#1}%
\auxpl{#2}{#3}{#4}}

\newcommand{\nextlinenumber}[1]%lineno
{\ifnum#1>\theplineno%
\setcounter{plineno}{#1}\addtocounter{plineno}{-1}%
\else\typeout{Illegal proofline-number #1 in ND-proof \ndprooflabel}%
\fi}

\newcommand{\planline}[2]%label,formula
{\pl{#1}{#1;}{#2}{PLAN;}}

\newcommand{\openline}[2]%label,assump,formula
{\pl{#1}{#1;}{#2}{Open;}}

\newcommand{\thmline}[3]%assump,formula,reason
{\auxthmline{#1}{#2}{#3}}

\newcommand{\pletc}%
{$\quad\vdots$\>$\quad\vdots$\>\>$\qquad\vdots$\>$\quad\vdots$\\}

\newlength{\boxbreite}\newlength{\strichbreite}
 
\newcommand{\ndproofdevider}[1]%
{\settowidth{\boxbreite}{ #1 }%
\setlength{\strichbreite}{0.5\textwidth}%
\addtolength{\strichbreite}{-0.5\boxbreite}%
\rule[0.8ex]{\strichbreite}{0.3pt} #1 \rule[0.8ex]{\strichbreite}{0.3pt}\\}

\newenvironment{ndproof}[1]%Beweislabel
{\global\def\ndprooflabel{#1}\setcounter{plineno}{0}
\footnotesize\begin{tabbing}%
\hspace*{1.1cm}\=\hspace*{2.5cm}\=\hspace*{0.6cm}%
\=\hspace*{9.5cm}\= \kill}%
{\end{tabbing}}

\newenvironment{ndproofsection}[1]%Beweislabel
{\footnotesize\begin{tabbing}%
\hspace*{1.1cm}\=\hspace*{2.5cm}\=\hspace*{0.6cm}%
\=\hspace*{9.5cm}\= \kill}%
{\end{tabbing}}

\newcommand{\ndproofheader}%
{No \> S;D \> \> Formula \> Reason\\[0.5ex]
\rule{\textwidth}{0.5pt}\\[0.5ex]}

\newcommand{\geru}[2]
{\begin{tabular}{c}{$#1$}\\ \hline{$#2$}\end{tabular}}
\newcommand{\gentzenrule}[3]
{{#1}\geru{#2}{#2}}
\newcommand{\gentzenrulem}[3]
{\gentzenrule{#1}{$#2$}{$#3$}}
\def\premdivider{{\;;\;\;}}

%%%%%%%%%%%%%%%%%%%%%%%%%%%%%%%%%%%%%%%%%%%%%%%%%%%%%%%%%%%%%%%%%%%%%%%%%%%%%%%
%                     Macros f"ur Methoden
%%%%%%%%%%%%%%%%%%%%%%%%%%%%%%%%%%%%%%%%%%%%%%%%%%%%%%%%%%%%%%%%%%%%%%%%%%%%%%%%
% Methoden sind strukturen mit 5 slots...
%
%%%%%%%%%%%%%%%%%%%%%%%%%%%%%%%%%%%%%%%%%%%%%%%%%%%%%%%%%%%%%%%%%%%%%%%%%%%%%%%

\def\printmethod{\begin{description}%
\advance\topsep by -5pt\advance \itemsep by -5pt
\item[name:]{\tt \methname}%
\item[status:]\methstatus%
\item[pre:]\preconditions%
\item[post:]\postconditions%
\item[Kb:]\knowlegeBit%
\item[proc:]\procedure%
\end{description}}

\newcommand{\methodname}[1]{\long\def\methname{#1}}%
\newcommand{\methodstatus}[1]{\long\def\methstatus{#1}}%
\newcommand{\precond}[1]{\long\def\preconditions{#1}}%
\newcommand{\postcond}[1]{\long\def\postconditions{#1}}%
\newcommand{\kb}[1]{\long\def\knowlegeBit{#1}}%
\newcommand{\proc}[1]{\long\def\procedure{#1}}%
\def\thmnl{\ \linebreak[4]}

\newenvironment{method}%
{\Defcase\begin{iAuxmethod}}%
{\thmnl\printmethod\end{iAuxmethod}}











