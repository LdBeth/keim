\chapter{Conventions for programming \keim}

Since {\sc Common Lisp} does not have a rigid type system the $\Omega$ group has enacted a standard for
programming \keim\ in order to keep up the quality of the code.
\section{Naming Conventions}
\label{naming-conventions}
\begin{enumerate}
\item Each module has a short name  {\tt <<Ext>>}, of 2-4 letters or digits
beginning with a letter.
This name is prefixed to all functions, defstructs, classes, generic
functions and variables.
\item Names for {\em interface-functions} are of the form  {\tt
<<Ext>>${}\sim{}$<<Name>>}. 
\item Names for {\em internal functions} are of the form {\tt
<<Ext>>=<<Name>>}, internal functions may only be used in the
corresponding module.
\item Functions, that are only used for debugging and testing the implementation
have names of the form {\tt <<Ext>>==<<Name>>}, these functions
must not be used by other functions.
\item {\tt internal variables} have names of the form  {\tt
<<Ext>>*<<Name>>}, note that there are no external variables.
\item Names for defined datatypes, such as CLOS defclasses, defstructs, and
 have the form {\tt <<Ext>>+<<Name>>} if they are to be exported and used 
in external modules and {\tt <<Ext>>$=$<<Name>>} if they are only for internal
use.
\item Names {\tt <<Name>>} are non-empty identifiers that
can be made from several parts composed
by the letter "{\tt -}"; the parts consist only of letters and digits: 

{\tt <<Name1>>-<<Name2>>-$\ldots$-<<Name$n$>>}. 
\item Predicates, functions that should be used only to return a Boolean value
of T or nil, should always be named with a {\tt -p} at the end, e.g., 
{\tt keim${}\sim{}$foo-p}.
\item Functions that are auxiliary functions (that is, they should never be called
directly, rather only through some top-level function), yet must be available
as generic functions in order for new modules to specialize on them, have
names of the following form: {\tt <<Ext>>$\sim=$<<Name>>}.
\item Property names (used on property lists of symbols or terms) should be of the form {\tt <<Ext>>$=$<<Name>>-PROP} for those internal to a module, or
{\tt <<Ext>>$\sim$<<Name>>-PROP} for those which are external.
\end{enumerate}

\section{Module Conventions}
\label{module-conventions}

Before a new module is generated it has to be discussed in the
\keim-group. If it is accepted the corresponding file-name is added to
the {\bf boot} file.  In the {\bf boot} file all files used in \keim\
are listed. They have to be generated with {\tt make-pathname}. The
sequence of the files corresponds to the sequence of loading the files.
Thereby it is assured that adaptions of pathnames to different
machines or Lisps dialects can be done by editing one single file.
Furthermore the whole system structure is mirrored in this file.

In order to get a simple and clearly structured system, the following
conventions must be observed:

\begin{itemize}
\item ``one file -- one module''. That is we have a one-to-one
correspondence of files and modules. If it is possible to split a
module in different files, it must be possible to generate different
modules.

\item The whole module structure is strictly hierarchical, that means
it has the form of a DAG. Thereby it is possible to load and compile
the whole system without any difficulties. The logical structure is
obvious.

\item The maximal size of module is 100k compiled code (comments are
not limited, therefore compiled code), compiled on Sparc under Lucid
with default compiler options. Thereby it is guaranteed that modules
are not too big, that they can be compiled with sufficiently small
expense, ...

\item In no file -- exept the boot file -- it is allowed to load
another file.

\item Every file must begin with {\tt (in-package "KEIM")}. In the
next line there must be a statement of the following form:
\begin{verbatim}
(mod~defmod A :uses (B C D)
              :documentation "This module is ...."
	      :exports (A~X A~Y))
\end{verbatim}

\item In every module {\tt A} only classes, interface functions,
internal functions or variables beginning with the module prefix (cp.\
naming conventions) may be defined ({\tt A+, A$\sim{}$, A*, A=}).
Furthermore it is possible to define methods for classes defined in an
imported module. That is, in the example above it is possible to
define:
\begin{verbatim}
(defmethod B~blabla ...)
\end{verbatim}

\item In every module {\tt A} variables and internal functions only of
this module may be used. In it may be used classes and interface
functions only of the module {\tt A} itself or modules imported by the
{\tt :uses}-command in the {\tt mod~defmod}-statement at the beginning
of the module.  Furthermore all standard \lisp-symbols can be used.

\end{itemize}

There will be \keim-functions for loading, compiling and loading the
system as well as for checking whether the conventions are observed.
The {\tt keim$\sim$-check-modules} function has to guarantee:

\begin{itemize}
\item On every module specified by the list of files in the boot file
there is an ``(in-package "KEIM")'' statement. On every module is a
{\tt mod~defmod} statement. Only the first is considered.

\item The module structure defined by the {\tt mod~defmod} statements
must be a DAG.

\item There is no {\tt load} statement in the modules. 

\item In every module only symbols begining with the module prefix are
defined. The only exception to this rule is the {\tt defmethod}-form,
which is needed to specialize the behaviour of methods inherited from
superclasses to those defined in the module. {\tt defmethod} may only
be used on generic functions from modules that the module explicity
imports.  

Whether a function is defined as an interface function or
as an internal function cannot be checked.

\item In every module only the allowed symbols are used.

\item The maximum size of modules is not checked.

\end{itemize}


\section{Documentation}
\begin{enumerate}
\item The {\tt defgeneric} form is supposed to contain all interface
      information (input, values effect,...) that is relevant to the user.
\item The {\tt defmethod} form should contain only internal
      information, all information relevant to the user should be in the
      {\tt defgeneric} form. the {\tt extract documentation} does not
      consider documentation in the {\tt defmethod} form.
\item For each method implicitly defined in the {\tt defclass} form there has to be a {\tt defgeneric}
      form with the documentation. Otherwise the accessor methods will not appear in the index 
      or the documentation.
\end{enumerate}

\section{Term Classes and Mixins}
{\keim} exports a hierarchy of general terms. From this the user can
tailor his own concept of terms by appropriate subclasses and mixins.
\begin{enumerate}
\item term classes are named {\tt term+term}, {\tt term+var}, {\tt term+const}, {\tt term+appl} and {\tt
term+abs} for generic (higher-order) terms
\item Each mixin (e.g. sorts, theories, \ldots) is a module in its own right and therefore has its 
      own short name {\tt <<Ext>>}.
\item The new classes are have the names {\tt term.<<Ext>>+term}, {\tt term.<<Ext>>+var},\ldots.
\item New functions have the forms {\tt term$\sim$<<Ext>>-<<Name>>}, {\tt term.<<Ext>>=<<Name>>},\ldots
Note that the generic functions {\tt term$\sim$<<Name>>} keep their name, even if the methods are 
specialized or even completely rewritten for the particular extension of the term class.
\end{enumerate}
