\chapter{Syntax of \protect{\post}}
\label{App:post-syntax}
\post\ ({\bf P}artial {\bf O}rder {\bf S}orted {\bf T}ype-Theory) is intended to be the integrating object
language for all \keim\ activities. At the moment \post\ is more powerful than the implementation that does
not yet support a sort concept. The current sort concept is that of well-typed terms, although the
introduction of arbitrary type constants corresponds to a simple many-sorted signature (i.e. there are no
subsort relations). In the unsorted version of \keim, the sort information coded into the sort
structure of \post\ will not be discarded, but will automatically be relativized (transformed) into a
(semantically equivalent) set of resulting (unsorted) terms. Thus only the existence of an order-sorted
language like \post\ can enhance the readability of the input and free the user of the (sometimes
error-prone) work of relativization.


{\bf $\Omega$-deduction problems} are lists, that contain
\begin{enumerate}
\item {\tt <Declarations>}
\item {\tt <Axioms>}
\item and an {\tt <Assertion>}
\end{enumerate}
The intended semantics of $\Omega$-deduction problems is that of a
theorem in mathematical practice: "Let {\tt <Declarations>}, then
{\tt <Assertion>}." In some theory defined earlier by {\tt <Axioms>}.\par
Here the implication {\tt <Axioms> $\models$ <Assertion>} is to be
validated in all models, that satisfy {\tt <Declarations>}.\par
\post\ is a variant of the simply typed $\lambda$-calculus, with
sorts, partial functions and modalities. In the following we give the
syntax of well-typed formulae in \post.

\section{Types}
\post\ formulae come with a type symbol, which specifies the type
(individual constant, function, predicate \ldots) of the object denoted
by the formula.\par
The {\bf base-types $\iota$} of {\bf individuals} and the type $o$ of {\bf truth values} are
denoted by the symbols {\tt I} and {\tt O} in \post.

Further base type symbols can be declared by {\tt (type-constants (T S ...)} and 
{\tt (type-variables (T S ...))}, where type variables used for polymorphic constants.

The syntax of type symbols is inductively defined by
\begin{enumerate}
\item type constants and type variables in the signature are types. In particular the predefined 
      base type symbols {\tt I} and {\tt O}  are types 
\item if ${\tt S}^1$,\ldots,${\tt S}^n$ and {\tt T} are type  symbols, then
{\tt (T ${\tt S}^1$\ldots${\tt S}^n$)} is also a type  symbol,
denoting the type of functions with $n$ arguments of types
 ${\tt S}^1$,\ldots,${\tt S}^n$ with the image in the sort {\tt T}.
\end{enumerate}

\section{Declarations}
\label{sec:declarations}
\subsection{The signature}
The signature is the datastructure that contains the type information for constants, therefore its
\post\ representation is a constant declaration of the form {\tt (constants <decs>)}, where
{\tt <decs>} is a sequence of declarations or the form {\tt (<symbol> <Type>)}, such that {\tt <symbol>} is a
new symbol and {\tt <Type>} is a Type.        

Since \post\ in theory expects closed formulae, all variables are bound and therefore are declared by the
binding operator {\tt LAM}. For the ease of notation there are {\bf variable declarations} of the form {\tt
(variables <vardecs>)} which are equivalent to a sequence of bindings by universal quantification. Since
\post\ is an order-sorted language, therefore variables are viewed to have a sort rather than a type, the
declaration {\tt <vardecs>} is a sequence of declarations or the form {\tt (<symbol> <Sort>)}, such that
{\tt <symbol>} is a new symbol and {\tt <Sort>} is a unary predicate. The type of a variables can be
computed from the sort of a variable. Therefore a variable declaration of the form {\tt (X S)} where S is a
term of type {\tt (0 T)} causes the variable {\tt X} to be declared of type {\tt T}.

In \post\ there are further {\bf term-} and {\bf subsort declarations} that carry further taxonomic
information. They are of the form {\tt (term-decs (<formula> <sort>))} and 
{\tt (subsort-decs (<sort> <sort>)} respectively. Here, the type of {\tt <formula>} has to be {\tt
T}, iff that of {\tt <sort>} is {\tt (O T)}, subsort declarations compare sort expressions of the
same type. These declarations will be supported operationally by future versions of \keim\ and are
relativized to new formulae which explicitly carry the sort information.


\section{Well-formed formulae}

Well-typed formulas have a type. Sorts are predicates (sets of objects of type {\tt T}), and therefore have the 
type of the form {\tt (O T)}.
\begin{enumerate}
\item If {\tt (X T)} is a declared variable, then {\tt X} is a
      well-typed formula of type {\tt T}.
\item If {\tt (c T)} is declared, then {\tt c} is a
      well-formed formula of type {\tt T}.
\item If the type of {\tt A} is ${\tt (T\;\;S}^1\ldots{\tt S}^n{\tt)}$ and for $i=1,\ldots,n$
      {\tt (type ${\tt B}^i$)=${\tt S}^i$},
      then {\tt (A ${\tt B}^1$ \ldots ${\tt B}^n$)} is a well-formed
      formula of type {\tt T}.
\item If the type of {\tt A} is {\tt T}, and ${\tt S}$ is a predicate of type {\tt (OR)}, then
      {\tt (LAM (X S) A)} and {\tt (LAM (X R) A)}  are well-formed formulae of type {\tt (T R)}.
\end{enumerate}
For a formal definition we give 
\section{A type inference System for \post}
This type inference systems is  terminating deduction system for assertions like 
"in some {\bf variable context} $\Gamma$ and some {\bf Signature} $\Sigma$ the formula (is well-typed and)
of the type $\alpha$". Variable contexts and signatures are of the form
\begin{description}
\item[Context] $\Gamma$ ::= $\emptyset \bigl|\Gamma\& {\tt X\colon\alpha}$,
\item[Signature] $\Sigma$ ::= $\emptyset \bigl|\Gamma\& {\tt c\colon\alpha}$,
\end{description}
where $\& F$ is the operation $\cup\{F\}$. In a slight abuse of notation we will write $X\in\Gamma$, if
$X\colon\alpha\in\Gamma$ for some $\alpha$.\par
We also need the operation of context enhancement
\[\Gamma\oplus X\colon\alpha\;\;\colon=\left\{
\begin{array}{cl}
(\Gamma\setminus\{X\colon\beta\})\& X\colon\alpha  & X\colon\beta\in\Gamma  \\
\Gamma\& X\colon\alpha                             & X\not\in\Gamma 
\end{array}\right.\]
We first give the type inference systems for \post\ terms. It is a simple variant of the ML-type system.
\def\gsv{{\Gamma;\Sigma\vdash\;\;}}
\begin{description}
\item[$\Gamma$-start] \geru{{\tt X:A}\in\Gamma}{\gsv{\tt X:A}}
\item[$\Sigma$-start] \geru{{\tt c:A}\in\Sigma}{\gsv {\tt c:A}}
\item[ftype]\geru{\gsv{\tt A:type}\premdivider{\tt B:type}}{\gsv{\tt(A\;\; B):type}}
\item[app]\geru{\Gamma\oplus\beta\colon{\tt type};\Sigma\vdash{\tt A\;\;:\;\;(}\alpha\beta^1...\beta^n{\tt)}
\premdivider{\tt B}^1\colon\beta^1\premdivider\ldots\premdivider{\tt B}^n\colon\beta^n}
{\gsv{\tt (A\;\;B}^1\ldots{\tt B}^n{\tt)\;\;:\;\;}\alpha}
\item[abs]\geru{\Gamma\oplus{\tt X:}\beta;\Sigma\vdash\;\;{\tt A:}\alpha\premdivider{\tt S:(O}\beta{\tt )}}%
{\gsv{\tt(LAM\;\;(X\;\;}\beta{\tt)\;\; A):(}\alpha\beta{\tt )}}
\item[rel]\geru{\Gamma\oplus\beta\colon{\tt type};\Sigma\vdash
{\tt(LAM\;\;(X\;\;}\beta{\tt) (ifthen\;\;(S\;\; X)\;\; A)) : }\alpha}
{\gsv{\tt(LAM\;\;(X\;\; S)\;\; A) : }\alpha}
\item[inst]\geru{\gsv{\tt A\;\;:\;\;}\tau}{\gsv{\tt A\;\;:\;\;}\sigma(\tau)}, where $\sigma$ is a type substituion 
with $\Dom(\sigma)\subset\Gamma$
\end{description}
As with the ML-Type system we have that each well formed formula has a unique least type, with respect to the 
subsumption ordering on types.\par

\subsection{Other \keim\ objects}

\begin{description}
\item[clauses] {\tt (clause <vardec> $L^1$ \ldots $L^n$)}, 
where the $L^i$ are literals, that is valid \post\ atoms and their negations and {\tt <vardec>} is a
variable declaration the of the form {\tt (variables ($X^1$ $S^1$) \ldots ($X^m$ $S^m$))} (cf.
\ref{sec:declarations}) and the $X^i$ are the free 
variables of sorts $S^i$ in the clause. The variables are declared in the clause, since clauses are
commonly kept variable-disjoint and therefore the variables are not known before the creation of the clause.

\item[substitutions] {\tt (substitution ($X^1$ \ldots $X^n$) ($T^1$ \ldots $T^n$))}, where the $X^i$ are the
domain and the $T^i$ specify the codomain of the substitution in question.

\item[positions] {\tt (position $k^1$ \ldots $k^n$)}, where the $k^i$ are natural numbers
\end{description}


\section{\post\ description of $\Omega$-deduction problems}

A \post\ problem consists of a set of {\bf input formulae} that represent the problem to be solved 
and a set of (partial) proofs and supplementary data in various formats. The status describes the 
problem as {\tt open}, {\tt proven}, {\tt transformed}, \ldots

\begin{verbatim}
<problem>       ::== (problem <name> <status> <inputs> <proof>*)
<status>        ::== (status <symbol>+)
<inputs>        ::== (inputs <inputdec>* <theorem>)
<inputdec>      ::== (<label> <dec>)
<inputdec>      ::== (<label> (axiom <name> <formula>))
<theorem>       ::== (<label> (theorem <name> <formula>))
<proof>         ::== <split>
<proof>         ::== <cnf>
<proof>         ::== <rp-proof>
<proof>         ::== <r-graph>
<proof>         ::== <nd-proof>
\end{verbatim}

\subsection{The Split protocol}

{\sc Das hier ist noch nicth genau durchdacht}

Preprocessing may enable the system to split the input problem into smaller problems which are 
simpler to solve. These problems are also in the 
\begin{verbatim}
<split>         ::== (split <problem> <problem>+ <split-just>)
<split-just>    ::== weiss ich noch nicht
\end{verbatim}

\subsection{ the CNF protocol}

{\sc Das hier ist noch nicht genau durchdacht}
\begin{verbatim}
<cnf>           ::== (cnf <name> <cnf-clauses> <cnf-relation>)
<cnf-clauses>   ::== (cnf-clauses <cnf-cl>+)
<cnf-cl>        ::== (<label> <clause>)
<cnf-cl>        ::== (<label> (skolemconstant <constdec>)
<cnf-relation>  ::== (cnf-relation <cnf-ref>+)
<cnf-rel>       ::== (<label> <rel-pair>)
<rel-pair>      ::== weiss ich noch nicht
\end{verbatim}

\subsection{\post\ description of resolution- paramodulation proofs}

A \post\ proof is a succession of numbered proof steps containing initial clauses, resolution,
paramodulation and factoring steps. All proof steps are represented in a uniform way:\par\noindent
{\tt ($\#$ $C$ $J$)}, where $\#$ is a number or a label for the proof step, $C$ is a clause and $J$ 
is a justification in one of the following forms.
\begin{description}
\item[resolution] {\tt (resolution $C$  ($\#p_1$ $\#L_1$) ($\#p_2$ $\#L_2$) $R$ $\sigma$)}, where $C$ is the
resulting clause, the $\#p_i$ are numbers of previous proof steps (denoting the parent clauses inside) and
the $\#L_i$ are the number of literal that has been resolved on in the resolution step. $R$ is the rule 
used for the resolution (like {\tt standard}, {\tt symmetric}, ...) and  $\sigma$ is the
unifier used in this resolution.
\item[paramodulation] {\tt (paramodulation $C$ ($\#p_1$ $\#L$) ($\#p_2$ $\#E$ $dir$) $R$ $\sigma$ $\Pi$)}, where $C$,
$\#p_i$, $L$ and $\sigma$ are as in the resolution above and $\#E$ is a literal in the clause in $\#p_2$, that
is an equation oriented in the direction $dir$, which has been applied to {\tt ($\#p_1$ $\#L$)} at 
the position $\Pi$. $dir$ is one of the keywords {\tt rl} (right to left) or {\tt lr} (left to right).
\item[factoring] {\tt (factor $C$ ($\#p$ $L_1$ $L_2$) $\sigma$)}
\end{description}
The representation of a resolution paramodulation proof will have the following BNF
\begin{verbatim}
<rp-proof>      ::==    (rp-proof <name> <proof-step>+)
<proof-step>    ::==    (<label> <clause> <step>)
<step>          ::==    (resolution <litlab> <litlab> <res-rule>
                               <substitution>)
<step>          ::==    (paramodulation <litlab> <elitlab> <res-rule>
                               <substitution> <position>)
<step>          ::==    (<label> <clause> (factor <substitution>))
<litlab>        ::==    (<label> <number>)
<elitlab>       ::==    (<label> <number> <dir>)
<dir>           ::==    lr
<dir>           ::==    rl
<res-rule>      ::==    {\em symbol}
\end{verbatim}

Here is an example for a refutation-paramodulation proof:\par

\begin{verbatim}
(problem FOO (status proven)
  (inputs
    (dec1 (constant (p (O I))))
    (dec2 (constant (q (O I I)))) 
    (dec3 (constant (a I)))
    (dec4 (constant (b I)))
    (dec5 (constant (c I)))
    (ax1 (axiom (exists (LAM (Z I) (forall (LAM (X I) 
                   (and (q a Z) (implies (p X) (= a Z)))))))))
    (ax2 (axiom (not (p c))))
    (thm (theorem (q a a))))
  (cnf FOO-cnf 
    (cnf-clauses 
      (in1 (clause (variables) (q a b)))
      (in2 (clause (variables (x I)) (= a b) (not (p x))))
      (in3 (clause (variables) (p c)))
      (in4 (clause (variables) (not (q a a)))))
    (cnf-relation
      (weiss ich noch nicht)))
  (rp-proof FOO-rp
    (d5 (clause (variables) (= a b)) 
            (resolution (in2 2) (in3 1) standard
                        (substitution (x) (c)))
    (d6 (clause (variables) (q a a))  
            (paramodulation (in1 1) (d5 1 rl) standard
                        (substitution () ()) (position 1 2)))
    (d7 (clause (varibles) emptyclause) 
            (resolution (d6 1) (in4 1) standard
                        (substitution () ())))))
\end{verbatim}

\subsection{Refutation graphs}
Deduction graphs consist of a set of clauses and polylinks, which consist of a set of literals in the set of
clauses, which are unifiable. Therefore polylinks are in \post\ in the form 
{\tt (polylink ($\#p_1$ $\#L_1$),\ldots,($\#p_n$ $\#L_n$) $\sigma$)}, where $\sigma$ is the witnessing unifier.
Note that polylinks always contain at least one positive and one negative literal.\par

\begin{verbatim}
<r-graph>       ::==    (r-graph <name> <polylink>*)
<polylink>      ::==    (polylink <litlab> <litlab>+
                            <substitution>)
\end{verbatim}

\subsection{\post\ descriptions of natural deduction proofs}

The representation of a natural deduction proof will have the following
BNF.

\begin{verbatim}
<nd-proof>     ::== (nd-proof <name> <proof-assumps> <conclusion> 
                      <plan-list> <line-list>)
<proof-assumps>   ::== (proof-hyps <formula>*)
<conclusion>   ::== (conclusion <formula>)
<plan-list>    ::== (plans <plan>*)
<plan>         ::== (<label>+)
<line-list>    ::== (lines <line>*)
<line>         ::== (<label> <hyps> <assertion> <justification>)
<hyps>         ::== (<label>*)
<justification>::== (<rule-name> (<term>*) (<label>*))
<rule-name>    ::== {\em symbol}
\end{verbatim}
with the following restrictions.  Every line must have a unique label.
The lines should appear in linearized order 
of the proof structure, such that no line's label should appear in another
line's hyps or justification before the line itself appears (not really 
necessary, but makes things easier).  Each member of the proof's
``plans'' attribute should be of the form 
\begin{verbatim}
(planned-line support-line*)
\end{verbatim}
where {\tt planned-line} should be the label of a line that is not yet 
justified, and each {\tt support-line} should be the label of a line from which the
unjustified line is to be justified.  

The assertion of the last line of the proof should be the same as the
proof's conclusion attribute.  Any hyps of the last line should have
assertions that are included in the proof's hyps slot.  

An unjustified line will have as justification {\tt (PLAN () ())},
while a hypothesis with label {\tt foo} 
will have the justification {\tt (HYP () (foo))}.  There should be
enough
type information in the formulas to be able to reparse them unambiguously.
Each proof rule will expect a certain format for its justification,
for example, ordering of the lines in the justification may be important.

Here's an example of a proof, assuming that I have the \post\ right.
\begin{verbatim}
(constants (P (O I)) (a I))

(nd-proof FOO
  (proof-hyps (all (lambda (x I) (P x))))
  (conclusion (P a))
  (plans (L1 L2))
  (lines 
    (L2 (L2) (all (lambda (x I) (P x))) (HYP () (L2)))
    (L1 (L2) (P a) (PLAN () ()))))
\end{verbatim}
