\section{The basic functionality of \keim}

The {\tt base-keim} system provides the basic functionality 
for all \keim\ objects. 

The {\tt keim} module defines the classes {\tt keim+object} which is 
a superclass of all other \keim\ classes
and {\tt keim+name} which is a superclass of all named \keim\ classes.  

It is a basic design decision, that all \keim\ objects can be written and read in plain ASCII format in a
common \post\ representation. \post\ is intended to be the lingua 
franca for all \keim\ applications. The
intention of \post\ is that it is easy to parse by recursive \lisp\ functions rather that to be  easily
readable by humans. Therefore \post\ syntax is close to \lisp\ syntax so 
that it can be read by
the \lisp\ reader.

\keim\ provides a facility that allows to attach on-line help 
information to all \keim\ objects. This
functionality is is provided by the {\tt help} module.

The {\tt env} module provides an abstract possibility to keep track of 
objects by their names or other \lisp\
keys. One prominent example for this is the association of 
term variables and contants with their name.

