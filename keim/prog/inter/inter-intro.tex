\section{Introduction}

With many systems, we will want to have some kind of interaction with a
user.  The modules contained in this system provide support for such a
user interface.

One thing we may like to have is the ability to define commands for the
user to apply.  We would prefer to do this in a controlled way, so that
we can check the types of arguments before we actually call the \lisp\ 
code that carries out the command.  For this reason, we define what are
called {\em argument types\/} in the module {\tt ARG} ({\tt argtypes.lisp}),
as well as some common argument types in the module {\tt ARGDEFS}
({\tt argdefs.lisp}). 

With argument types defined, we can define commands. These act as the
interface between what the user inputs and the actual \lisp\ functions
that do that job.

We would also like to be able to have our system run with various types
of interface systems, ASCII terminals being just one of them.  We want
to write our programs without having to know just what kind of interface
they'll be used on, so that we can add and modify interfaces without
making our already-written programs obsolete.  In the module {\tt INTER}
({\tt inter.lisp}), we define a general framework for user interfaces, and
in the module {\tt ASI} ({\tt ascii-interface.lisp}), we define a
particular instance for ASCII terminals.

