\section{Introduction}

This will be the introduction to the \keim\  facility for proofs in the 
natural deduction format.

There are two modules in this system.  The first, called {\tt ND} and found in 
{\tt nd.lisp}, defines the basic functionality of the natural deduction
class of objects.   

Natural deduction proofs are represented as a collection of lines, each line
having an associated formula.  Each line has as well a list of hypotheses
from which it is logically dependent, in addition to a justification. 
A justification may still be {\em planned\/}, indicating that
the line has yet to be {\em proven}.  In a line which has been proven,
this justification will show through which inference rule and from which
preceding lines the line was derived.

The module {\tt RULE}, found in {\tt rule.lisp}, defines how inference rules
can be defined and used.  An inference rule defines a schema which can be
match to existing lines in a proof and cause the creation of new lines
and/or the justification of old lines. 

There are some sample rule definitions to be found in the file {\tt
rule-examples.lisp}.

Our natural deduction proofs are greatly influenced by those used in the
TPS system from Peter Andrews. 


