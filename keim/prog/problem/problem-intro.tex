\section{\keim\ problems}

This will be the introduction to the \keim\ problem facility.

This \keim\ system provides the basic functionality for defining and manipulating proof objects. The key
purpose for all classes and interface functions is to provide a common basis for various proof formats and
calculi.  The common structure supported in this system is that proofs are trees (or rather directed acyclic
graphs or DAGs) where the {\em nodes} are objects that contain a formula and a {\em justification}, which in turn
contain a rule and a set of supporting nodes. These dependencies give the DAG structure of the proof. 

Since proofs only make sense in a certain context \keim\ also provides the class of {\em problem} objects,
which record the curent environment, the assumptions and the conclusion that is proven in the proof. 
\keim\ problems also make sense without a proof.

